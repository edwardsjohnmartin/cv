\documentclass[margin,line]{res}

\usepackage{version}
\usepackage{versions}
\usepackage{url}
\usepackage{booktabs}
\usepackage{enumitem}
\usepackage[usenames,dvipsnames]{color}

\setlength{\pdfpagewidth}{\paperwidth}
\setlength{\pdfpageheight}{\paperheight} 

\oddsidemargin -.5in
\evensidemargin -.5in
%\topmargin-2cm
\textwidth=6.0in
\textheight23.5cm
\itemsep=0in
\parsep=0in

\newenvironment{list1}{
  \begin{list}{\ding{113}}{%
      \setlength{\itemsep}{0in}
      \setlength{\parsep}{0in} \setlength{\parskip}{0in}
      \setlength{\topsep}{0in} \setlength{\partopsep}{0in} 
      \setlength{\leftmargin}{0.17in}}}{\end{list}}
\newenvironment{list2}{
  \begin{list}{$\bullet$}{%
      \setlength{\itemsep}{0in}
      \setlength{\parsep}{0in} \setlength{\parskip}{0in}
      \setlength{\topsep}{0in} \setlength{\partopsep}{0in} 
      \setlength{\leftmargin}{0.2in}}}{\end{list}}

%\newcommand{\pubunder}[1]{\underline{#1}}
\newcommand{\pubunder}[1]{#1}

%-------------------------------------------------------------------------------
% this makes list spacing much better.
%-------------------------------------------------------------------------------
\newenvironment{tightenumerate}{
\begin{enumerate}
  \setlength{\itemsep}{1pt}
  \setlength{\parskip}{0pt}
  \setlength{\parsep}{0pt}
}{\end{enumerate}
}

%-------------------------------------------------------------------------------
% this makes list spacing much better.
%-------------------------------------------------------------------------------
\newenvironment{tightitemize}{
\begin{itemize}
  \setlength{\itemsep}{1pt}
  \setlength{\parskip}{0pt}
  \setlength{\parsep}{0pt}
}{\end{itemize}
}

\includeversion{LONG} 
%\excludeversion{LONG} 

%\includeversion{IMPACT}
\excludeversion{IMPACT}

\includeversion{PENDING}
%\excludeversion{PENDING}

\begin{document}

\name{\Large{John Edwards, Ph.D.} \vspace*{.1in}}

\begin{resume}
\section{\sc Contact Information}
\vspace{.05in}
%\begin{tabular}{@{}p{3in}p{3in}}
%\begin{tabular}{@{}p{3.7in}p{3in}}
%Center for Extreme Data Management and Visualization & \\
%Scientific Computing and Imaging Institute   &  \\%801.585.3911 (office)  \\ 
%University of Utah      &  (385) 207-8331 \\
%WEB 4660          &  jedwards@sci.utah.edu \\
%Salt Lake City, UT 84112                &  http://sci.utah.edu/$\sim$jedwards
%\end{tabular}

%\begin{tabular}{@{}p{3.7in}p{3in}}
%Center for Extreme Data Management and Visualization & (385) 207-8331 \\
%Scientific Computing and Imaging Institute   & jedwards@sci.utah.edu \\
%University of Utah      &  http://sci.utah.edu/$\sim$jedwards \\
%WEB 4660          &  \\
%Salt Lake City, UT 84112                & 
%\end{tabular}

\begin{tabular}{@{}p{3.7in}p{3in}}
Department of Informatics and Computer Science & (385) 207-8331 \\
Idaho State University   & edwardsjohnmartin@gmail.com \\
BA 337      &  www2.cose.isu.edu/$\sim$edwajohn \\
Pocatello, ID 83209          &
\end{tabular}

%\section{\sc Research Interests}
\section{\sc Research}

Geometric modeling, simulation, scientific visualization

\section{\sc Education}

%\textbf{Ph.D.} Computer Science, The University of Texas, 2013\\
%\textbf{M.S.} Computer Science, Brigham Young University, 2004\\
%\textbf{B.S.} Computer Science, Utah State University, 1998\\
Ph.D. Computer Science, The University of Texas, 2013\\
M.S. Computer Science, Brigham Young University, 2004\\
B.S. Computer Science, Utah State University, 1998\\

\section{\sc Professional experience}

Assistant professor, Utah State University, 2018-current\\
Assistant professor, Idaho State University, 2015-2018\\
Post-doctoral fellow, Scientific Computing and Imaging
Institute, University of Utah, 2013-2015\\
Visiting scholar, University of Hong Kong, 2012\\
Assistant instructor (during PhD studies), The University of Texas, 2010-2013\\
%Teaching assistant, Computer Science Department, The University of Texas, 2009-2012\\
Robotics and visualization research engineer, Autonomous Solutions, Inc., 2008-2009\\
Research and development engineer, ProLogic, Inc., 2005-2008 \\
Software engineer, Rigaku, Inc., 1999-2005 \\

%\section{\sc Honors}
%
%% Professional Development Awards for travel to Israel ($450), Rhode
%% Island ($250) and San Jose ($275) for SPM, BioVis and IMR, respectively.
%%The University of Texas, Professional Development Awards, 2010, 2011, 2012 \\
%% $1000 fellowship for first year PhD studies
%The University of Texas, Computer Science PhD Fellowship, 2009 \\
%Graduation \emph{Magna Cum Laude}, Utah State University, 1998 \\
%Member \emph{Phi Kappa Phi} Honor Society, 1998 \\
%Wendell Pope Scholarship, Utah State University, 1998 \\
%Superior Student Scholarship, Utah State University, 1996-1998 \\

%\section{\sc Peer-reviewed publications}
%\vspace{1.2cm}
\section{\sc Journal publications}

DeWayne Derryberry, Ken Aho, \pubunder{John Edwards}, and Teri Peterson. Model selection and regression t-statistics. \textit{The American Statistician}. In press. 2018.
\begin{IMPACT}
Scopus impact factor: 0.51 % scopus
%SCI impact factor: 
\end{IMPACT}

Nathan Morrical$^*$ and \pubunder{John Edwards}. Parallel quadtree construction on collections of objects. \textit{Computers and Graphics}. 66:162–168. 2017.
\begin{IMPACT}
Scopus impact factor: 1.63 % scopus
%SCI impact factor: 
\end{IMPACT}
\\\begin{footnotesize}$^*$ Undergraduate student author\end{footnotesize}

Boyd Edwards and \pubunder{John Edwards}. Periodic nonlinear sliding modes for two uniformly magnetized spheres. \textit{Chaos: An Interdisciplinary Journal of Nonlinear Science}. 27(5):053107, 2017.
\begin{IMPACT}
Scopus impact factor: 1.76 % scopus
%SCI impact factor: 2.283
\end{IMPACT}

Boyd Edwards and \pubunder{John Edwards}. Dynamical interactions between two uniformly magnetized spheres. \textit{European Journal of Physics}. 38(1):015205, 2016.
\begin{IMPACT}
Scopus impact factor: 0.54 % scopus
%SCI impact factor: 0.614
\end{IMPACT}

Xin Tong, \pubunder{John Edwards}, Chun-Ming Chen, Han-Wei Shen, Christopher Johnson, and Pak Chung Wong. View-dependent streamline deformation and exploration. \textit{IEEE Transactions on Visualization and Computer Graphics}. 22(7):1788-1801, 2016.
\begin{IMPACT}
Scopus impact factor: 3.94 % scopus
%SCI impact factor: 2.84
\end{IMPACT}

\pubunder{John Edwards}, Eric Daniel, Valerio Pascucci, Chandrajit Bajaj. Approximating the Generalized Voronoi Diagram of Closely Spaced Objects. \textit{Computer Graphics Forum}. 34(2):299-309, 2015.
\begin{IMPACT}
Scopus impact factor: 2.33 % scopus
%SCI impact factor: 1.542
\end{IMPACT}

\pubunder{John Edwards}, Eric Daniel, Justin Kinney, Terrence Sejnowski, Tom Bartol, Daniel Johnston, Kristen Harris, and Chandrajit Bajaj. VolRoverN: Enhancing surface and volumetric reconstruction for realistic dynamical simulation of cellular and subcellular function.  \textit{Neuroinformatics}. 12(2):277-289, 2014.
\begin{IMPACT}
Scopus impact factor: 2.72 % scopus
%SCI impact factor: 2.825
\end{IMPACT}

\pubunder{John Edwards} and Chandrajit Bajaj. Topologically correct reconstruction of tortuous contour forests. \textit{Computer-Aided Design}. 43(10):1296-1306, 2011.
\begin{IMPACT}
Scopus impact factor: 2.57 % scopus
%SCI impact factor: 
\end{IMPACT}

\begin{IMPACT}
% * Scopus impact factors used
\end{IMPACT}

\newpage

\section{\sc Refereed conference publications}

John Edwards, Erika Fulton, Jonathan Holmes, Joseph Valentin$^\ddagger$, David Beard, and Kevin Parker. Separation of syntax and problem solving in Introductory Computer Programming. \textit{IEEE Frontiers in Education.} San Jose, CA. October 2018.

Sidharth Kumar, Duong Hoang, Steve Petruzza, Valerio Pascucci, and \pubunder{John Edwards}. Reducing network congestion and synchronization overhead during data aggregation when writing hierarchical data. \textit{IEEE International Conference on High Performance Computing, Data, and Analytics.} Jaipur, India. December 2017. 23\% acceptance rate.
%\begin{IMPACT}
%23\% acceptance rate.
%\end{IMPACT}

Nathan Morrical$^*$ and \pubunder{John Edwards}. Parallel quadtree construction on collections of objects. \textit{Shape Modeling International}. Berkeley, CA. June 2017. 37\% acceptance rate.
$^\dagger$
%\begin{IMPACT}
%37\% acceptance rate.
%\end{IMPACT}
%18/49 papers accepted to SMI

\pubunder{John Edwards}, Eric Daniel, Valerio Pascucci, Chandrajit Bajaj. Approximating the Generalized Voronoi Diagram of Closely Spaced Objects. \textit{Eurographics}. Zurich, Switzerland. July 2015. 27\% acceptance rate.
$^\dagger$
%\begin{IMPACT}
%27\% acceptance rate.
%\end{IMPACT}

Sidharth Kumar, \pubunder{John Edwards}, Peer-Timo Bremer, Aaron Knoll, Cameron Christensen, Venkatram Vishwanath, Philip Carns, John A. Schmidt, Valerio Pascucci. Efficient I/O and storage of adaptive resolution data. \textit{High Performance Computing, Networking, Storage and Analysis (SC14)}. New Orleans, LA. November 2014. 21\% acceptance rate.
%\begin{IMPACT}
%21\% acceptance rate.
%\end{IMPACT}

\pubunder{John Edwards}, Wenping Wang, and Chandrajit Bajaj. Surface segmentation for improved remeshing. \textit{Proceedings of the 21st International Meshing Roundtable}, pages 403-418. San Jose, CA. October 2012.

\pubunder{John Edwards} and Chandrajit Bajaj. Topologically correct reconstruction of tortuous contour forests. \textit{Proceedings of the ACM Symposium on Solid and Physical Modeling}, pages 51-60. Haifa, Israel. September 2010. 29\% acceptance rate.
%\begin{IMPACT}
%29\% acceptance rate.
%% 51 submitted, 15 full papers accepted (11 short papers accepted) for 29\% acceptance rate
%\end{IMPACT}

Joel Alberts, \pubunder{John Edwards}, Josh Johnston, and Jeff Ferrin. 3D visualization for improved manipulation and mobility in EOD and combat engineering applications. \textit{Proceedings of SPIE Defense, Security and Sensing}. April 2009.

Josh Johnston, Joel Alberts, Matt Berkemeier, and \pubunder{John Edwards}. Manipulator Autonomy for EOD Robots. \textit{26th Army Science Conference}. December 2008.
%% <www.edwardssoftware.com/john/ManipulatorAutonomy.pdf>

%John Edwards. LiveMesh: An Interactive 3D Image Segmentation Tool. Masters Thesis, December 2004.
%% <www.edwardssoftware.com/john/Thesis.pdf>

\begin{footnotesize}$^\dagger$ Also listed under Journals section\end{footnotesize}

\section{\sc Book chapter}

John Edwards, Sidharth Kumar, and Valerio Pascucci. Big data from scientific simulations. In L. Grandinetti, G.R. Joubert, M. Kunze, and V. Pascucci, editors, \textit{Big Data and High Performance Computing}, pages 32--46. IOS Press, Amsterdam, Berlin, Tokyo, Washington, DC, 2015.

\begin{PENDING}
\section{\sc Pending publications}

Lloyd Griffel$^\ddagger$, Donna Delparte, and John Edwards. A machine learning approach using spectral signatures to detect potato plants infected with Potato Virus Y. \textit{Submitted.} 2018.
\\\begin{footnotesize}$^\ddagger$ Student author\end{footnotesize}
\end{PENDING}

%\section{\sc Submitted}

%John Edwards, Eric Daniel, Chandrajit Bajaj, Valerio Pascucci. The Generalized Voronoi Diagram of Closely Spaced Objects. \textit{Submitted}. 2014.

\section{\sc Grants funded}
%\begin{itemize}[label={},leftmargin=0pt]
%  \setlength\itemsep{0.5em}
%\item Won: \$184,017\\Pending: \$248,304
%\item Won: \$184,017 (\$22,842 as PI)\\Pending: \$1,135,880 (\$235,740 as PI)
%\item Won: \$184K (\$23K as PI)\\Pending: \$1.1m (\$236K as PI)
%\section{\sc Funded (\$184,017)}
%\item \underline{Funded (\$184,017)}
\textit{Syntax before problem solving: an approach to introductory computer programming education}. PI: \underline{J. Edwards}. co-PIs: E. Fulton, J. Holmes, D. Beard, K. Parker. Idaho State University Office of Research. \$32,564. 2017. Funded.

\textit{Implementing Unmanned Aircraft Systems to detect crop viruses using hyperspectral remote sensing and machine learning}. PI: D. Delparte. co-PI: \underline{J. Edwards}. Idaho State Dept. of Agriculture. \$161,175. 2017. Funded.

\textit{Unified modeling and visualization of avalanche flow paths}. PI: \underline{J. Edwards} with S. Pawlidis (student). STEM Undergraduate Research Initiative, Idaho SBOE. \$1,740. 2017. Funded.

\textit{ISU CoSE Internal Travel Grant}. \underline{J. Edwards}. \$2,000. 2017. Funded.

\textit{Improving STEM Education: Engaged Learning in an Introductory  Computer Programming Course}. PI: \underline{J. Edwards}. co-PIs: J. Holmes, K. Parker. ISU Teaching Innovation Grant. \$4820. 2017. Funded.

\textit{STEM Action Center: Computer Programming Workshops in Southeastern Idaho}. C. Hill, et al. United Way. \$8000. Role: Senior personnel. 2017. Funded.

\textit{igniteCS: CS education in Southeastern Idaho high schools}. PI: \underline{J. Edwards} co-PIs: J. Rose, J. Glines, et al. Google igniteCS gift. \$5307. 2016. Funded.

\textit{NIH BRAIN Initiative Summer Course on interdisciplinary computational neuroscience}. \underline{J. Edwards}. Competitive admission to funded workshop at University of Missouri. 31\% acceptance rate. 2016. Funded. % 31% acceptance rate (24/77)

\textit{UTexas Professional Development Award for travel to San Jose, CA}. \underline{J. Edwards}. \$275. 2012. Funded. %(\$275)

\textit{UTexas Professional Development Award for travel to Providence, RI}. \underline{J. Edwards}. \$250. 2011. Funded. %(\$275)

\textit{UTexas Professional Development Award for travel to Haifa, Israel}. \underline{J. Edwards}. \$450. 2010. Funded. %(\$275)
%\end{itemize}

\section{\sc Grants pending}
%\begin{itemize}[label={},leftmargin=0pt]
%  \setlength\itemsep{0.5em}
%\item \underline{Pending (\$248,304)}
\textit{Web-Based Simulations for Intermediate Mechanics Education}. PI: \underline{J. Edwards}. co-PI: S. Shropshire. NSF 17-561. \$215,740. 2017. Pending.
%\end{itemize}

\section{\sc Grants not funded}
%\begin{itemize}[label={},leftmargin=0pt]
%  \setlength\itemsep{0.5em}
%\item \underline{Not funded}
\textit{In-situ plant virus detection system using advanced remote sensing and machine learning}. PI: D. Delparte. co-PI: \underline{J. Edwards}. NSF 17-529. \$900,140. 2017. Not funded.

\textit{Sensor Systems for Decision Support in the Food, Energy and Water Nexus - Testbed in Idaho's Snake River Plain}. PI: D. Delparte. co-PIs: C. Baxter, S. Godsey, F. Harris, D. Van Horn J. Rachlow, J. Forbey, \underline{J. Edwards}. NSF15-517. \$5,999,723. 2015. Not funded. %(Co-PI) % NSF 15-517 is the solicitation. \$5,999,723

\textit{iSEED: UAS to Advance SES Modeling.}. PI: J. Johnston. co-PIs: J. Forbey, L. Vierling, J. Eitel, \underline{J. Edwards}, D. Delparte. NSF EPSCoR Managing Idaho’s Landscapes for Ecosystem Services (MILES) iSEED. \$242,976. 2015. Not funded. % \$242,976.07

\textit{MRI: Acquisition of advanced Unmanned Aircraft Systems (UAS) remote imaging sensors and ground-based spectrometry for understanding structural and phytochemical function of ecosystems.}. PI: D. Delparte. co-PIs: \underline{J. Edwards}, J. Rachlow, J. Forbey. NSF15-504. 2015. Not funded. % 15-504 is the solicitation

\textit{Morse Smale Crystals as Meshing Primitives for Lagrangian Simulations}. PI: P-T. Bremer. co-PIs: V. Pascucci, \underline{J. Edwards}, A. Gyulassy. DOE LAB 14-1003. 2013. Not funded.

%\item \textit{Form and Function: Multiscale Modeling of Electrophysiology in the Hippocampus}. PI: C. Bajaj. co-PI: D. Johnston. NSF CRCNS 1311276. 2010. Role: Contributor. Not funded.
%\end{itemize}

\section{\sc Theses}
(Ph.D.) Analysis-Ready Models of Tortuous, Tightly Packed Geometries, 2013\\
(M.S.) Live Mesh: An Interactive 3D Image Segmentation Tool, 2004

%\section{\sc Thesis}
%Analysis-Ready Models of Tortuous, Tightly Packed Geometries, 2013\\

\section{\sc Research Software}

\emph{pgvd} - Parallel Generalized Voronoi Diagram Approximation\\
\url{https://github.com/edwardsjohnmartin/pgvd.git}

\emph{3PIO} - Powerful and Practical Program IDE Online\\
\url{https://github.com/edwardsjohnmartin/3PIO.git}

\emph{avalanche} - Avalanche simulation software\\
\url{https://github.com/edwardsjohnmartin/avalanche.git}

\emph{MagPhyx} - Magnet simulation software\\
\url{http://www2.cose.isu.edu/~edwajohn/MagPhyx}

\emph{GVD} - Generalized Voronoi Diagram approximation\\
\url{http://www2.cose.isu.edu/~edwajohn/research/gvd}

\emph{VolRoverN} - Neuronal reconstruction and geometric analysis\\
http://www.cs.utexas.edu/$\sim$bajaj/cvcwp/?page\_id=2089

%\emph{SimpleSeg} - Image segmentation in 3D using LiveMesh (no longer maintained)\\
%http://www.youtube.com/watch?v=9KfhFvIvFK4
%http://www.edwardssoftware.com/john

\begin{LONG}

\section{\sc Posters}
%\section{\sc Posters\\$\star$ Best poster award}
J. Ory, W. Grigg, J. Edwards, J. Holmes, K. Parker.
3PIO: Powerful and Practical Python IDE Online
\textit{Idaho Conference on Undergraduate Research}. Boise, ID, July 2017.

W. Grigg, S. Denton, J. Edwards, J. Stover.
Hidden associations: visualizing word-to-word connections in Tweets
\textit{Idaho Conference on Undergraduate Research}. Boise, ID, July 2017.

J. Valentin, K. Aho, J. Edwards, D. Derryberry, T. Peterson.
Improving computational efficiency in identifying parsimonious statistical models
\textit{Idaho Conference on Undergraduate Research}. Boise, ID, July 2017.

G. Cochrane, J. Edwards, D. Delparte.
LiDAR Odometry and Mapping for Terrain Analysis from Unmanned Aerial Vehicles
\textit{Idaho Conference on Undergraduate Research}. Boise, ID, July 2017.

G. Cochrane, M. Sterbentz, J. Edwards.
Real-Time LiDAR Terrain Mapping and Analysis
\textit{Idaho Conference on Undergraduate Research}. Boise, ID, July 2016.

J. Glines, J. Edwards.
Isosurface Extyraction in a Simple C/C++ Library
\textit{Idaho Conference on Undergraduate Research}. Boise, ID, July 2016.

N. Vollmer, N. Harrison, J. Edwards.
An Adaptive, Parallel Algorithm for Approximating the Generalized Voronoi Diagram
\textit{Idaho Conference on Undergraduate Research}. Boise, ID, July 2016.

J. Edwards, C. Johnson.
Visualizing white matter tracts in the human brain.
\textit{SIAM Conference on Computational Science and Engineering}. Salt Lake City, UT, March 2015.

S. Kumar, B. Summa, C. Christensen, J. Edwards, V. Pascucci.
Multi-resolution I/O for Massive Simulations: Enabling Scalable Visualization and Processing.
\textit{Predictive Science Academic Alliance Program (PSAAP) TST Meeting}. Palo Alto, CA, May 2014.

J. Edwards, E. Daniel, C. Bajaj, J. Kinney, T. Bartol, T. Sejnowski, K. Harris, D. Johnston.
VolumeRoverN: Analysis-ready domain models of neuronal forests.
\textit{2nd Annual Austin Translational Neuroscience Symposium}. Austin, TX, October 2012. \\
$\star$ Best poster award

%J. Edwards, J. Kinney, K. Harris, C. Bajaj. 
%High quality 3D geometric models of hippocampal neuropil for electrophysiological simulation. 
%\textit{Janelia Farm Conference on Large-Scale 3D Image Annotation, Management, and Visualization}. Ashburn, VA, October 2012.

J. Edwards, A. Rand, J. Kinney, K. Harris, C. Bajaj. 
Analysis-ready meshes of neuronal forests. 
\textit{1st IEEE Symposium on Biological Data Visualization}. Providence, RI, October 2011.

J. Edwards, A. Gillette, R. K. Bettadapura, A. Rand, C. Rumsey, Q. Zhang, D. Johnston, K. Harris, C. Bajaj. 
Electrophysiological Models Derived from EM Reconstructions. 
\textit{The National Academies Keck Futures Initiative Conference on Imaging Science}. November 2010.

A. Gillette, R. K. Bettadapura, F. Chowdury, J. Edwards, A. Gopinath, J. Rivera, B. Subramanian, A. Rand, C. Rumsey, Q. Zhang, D. Johnston, K. Harris, C. Bajaj, T. Bartol, D. Keller, J. Kinney, T. Sejnowski. 
Spatially Realistic and Reduced Electrophysiology Models Derived From EM Reconstruction. 
\textit{MPG-HHMI Janelia Farm High-Resolution Circuit Reconstruction Conference}. Berlin, Germany, September 2009.


\section{\sc Contributed talks}
MagPhyx: simulation and visualization of magnet dynamics.
\textit{Idaho Academy of Science and Engineering}. Pocatello, ID. April 1, 2016.

Surface segmentation for improved isotropic remeshing.
\textit{21st International Meshing Roundtable}. San Jose, CA. Oct 9, 2012.

Topologically correct reconstruction of tortuous contour forests.
\textit{14th ACM Symposium on Solid and Physical Modeling}. Haifa, Israel. September 1, 2010.

Advanced techniques for LiDAR visualization and analysis using ArcGIS.
\textit{8th International Lidar Mapping Forum}. Denver, CO. February 2008.

%\newpage

\section{\sc Invited talks}
At the whiteboard: collaborative data science projects
\textit{Utah State University}. Logan, UT. Dec 8, 2017.

The adventure of discovery: geometry, topology and visualization.
\textit{Idaho State University}. Pocatello, ID. Oct 17, 2014.

Exploration of high-dimensional scalar functions.
\textit{Computational Visualization Center group meeting}. Austin, TX. Nov 13, 2013.

Analysis-ready models of tortuous, tightly packed geometries.
\textit{University of Colorado Medical Center}. Denver, CO. Mar 1, 2013.

Analysis-ready models of tortuous, tightly packed geometries.
\textit{New Mexico State University}. Las Cruces, NM. Feb 15, 2013.

Analysis-ready models of tortuous, tightly packed geometries.
\textit{Scientific Computing and Imaging Institute}. Salt Lake City,
UT. Feb 8, 2013.

Cool Geometry Stuff.
\textit{Leander High School, Anna Bouboulis Geometry Class}. Leander, TX. Jan 5, 2012.

Surface segmentation for improved isotropic remeshing.
\textit{University of Hong Kong graphics group meeting}. Hong Kong. May 30, 2012.

Polyhedron separation.
\textit{Computational Visualization Center group meeting}. Austin, TX. Sept 7, 2011.

Analysis-ready 3D reconstructions of complex objects from planar cross-sectional slices.
\textit{Computational Visualization Center group meeting}. Austin, TX. Mar 25, 2011.

The connectome: challenges and approaches.
\textit{Computational Visualization Center group meeting}. Austin, TX, Oct 27, 2010.

LidarExplorer
\textit{Advanced LiDAR Workshop at the GeoTREE Center of the University of Northern Iowa}. August 2007.

%\section{\sc Other talks}
%Analysis-ready models of tortuous, tightly packed geometries.
%\textit{Thesis Defense}. Austin, TX, May 20, 2013.
%
%Analysis-ready domain models of neuronal forests.
%\textit{Thesis proposal}. Austin, TX, Dec 13, 2011.
%
%Analysis-ready 3D reconstructions of complex objects from planar cross-sectional slices.
%\textit{Research Preparation Exam}. Austin, TX, March 29, 2011.

%\newpage

\end{LONG}

\section{\sc Conferences}
\textit{ACM Special Interest Group for Computer Science Education (SIGCSE).} Baltimore, MD. February 2018.

\textit{Shape Modeling International.} Berkeley, CA. June 2017.

\textit{Idaho EPSCoR Annual Meeting.} Coeur d'Alene, ID. October 2016.

\textit{Idaho Conference on Undergraduate Research.} Boise, ID. July 2016.

\textit{Idaho Academy of Science and Engineering}. Pocatello, ID. April 2016.

\textit{Eurographics.} Zurich, Switzerland. July 2015.

\textit{High Performance Computing, Networking, Storage and Analysis (SC14).} New Orleans, LA. November 2014.

\textit{{IEEE VIS}.} Atlanta, GA. October 2013.

\textit{21st International Meshing Roundtable.} San Jose, CA. October 2012.

\textit{Austin Translational Neuroscience Symposium.} Austin, TX. October 2012.

\textit{IEEE Symposium on Biological Data Visualization.} Providence, RI. October 2011.

\textit{ACM Symposium on Solid and Physical Modeling.} Haifa, Israel. September 2010.

\section{\sc Workshops}

\textit{Developing Empirical Education Research Studies (DEERS).} Charlottesville, VA. July 17-19, 2018.

\textit{Deep Learning in the Classroom.} SIGCSE. February 23, 2018.
% Doug Blank (Bryn Mawr), Lisa Meeden (Swarthmore), Jim Marshall (Sarah Lawrence)

\textit{Designing Empirical Education Research Studies (DEERS): Creating an Answerable Research Question.} SIGCSE. February 21, 2018.
% Mark Sherriff (U. of Virginia)

\textit{Integrating Cloud Computing into the Computer Science Curriculum.} SIGCSE. February 21, 2018.
% Laurie White (Google)

\textit{NSF XSEDE Workshop on data modeling.} Boise State University. July 18-20, 2016.

\textit{NIH BRAIN Initiative Summer Course on interdisciplinary computational neuroscience.} University of Missouri. June 5-17, 2016.

\textit{Grant Writers Workshop.} Idaho State University. February 29, 2016.

\textit{Promotion and Tenure Workshop.} Idaho State University. November 18, 2015.

\textit{Interactive Cooperative Grant Training.} Idaho State University. August 31, 2015.

\textit{Advanced LiDAR Workshop}. University of Northern Iowa. August 2007.

%\section{\sc Awards}
\section{\sc Honors}
\begin{itemize}[label={},leftmargin=0pt]
  \setlength\itemsep{0em}
  \item Translational Neuroscience Symposium Best Poster Award, 2012 %(\$200)
  \item The University of Texas, Computer Science PhD Fellowship, 2009 %(\$1000)
  \item Graduation \emph{Magna Cum Laude}, Utah State University, 1998
  \item Member \emph{Phi Kappa Phi} Honor Society, 1998
  \item Wendell Pope Scholarship, Utah State University, 1998
  \item Superior Student Scholarship, Utah State University, 1996-1998
\end{itemize}

\begin{LONG}
%\newpage
\end{LONG}

\section{\sc Professional service}
Program committee
\vspace{2mm}
\begin{itemize}[label={},leftmargin=5mm]
  \item International Conference on Geometric Modeling and Processing (GMP) 2015, 2016, 2017, 2018
\end{itemize}

Reviewer
\vspace{2mm}
\begin{itemize}[label={},leftmargin=5mm]
  \item ACM Transactions on Mathematical Software
  \item GMP 2015, 2016, 2017, 2018
  \item Computing Surveys
  \item Computer Aided Geometric Design
  \item European Symposium on Algorithms 2014
  \item International Meshing Roundtable 2015
  \item SIGGRAPH Asia 2015
\end{itemize}

\section{\sc University service}
%\vspace{2mm}
\begin{itemize}[label={},leftmargin=0mm]
  \setlength\itemsep{0em}
  \item Chair of CS faculty search committee 2017-2018
  \item University Research Council 2016-2018
  \item Health Informatics search committee 2016
  \item Author of CS Masters Degree proposal submitted 2018
\end{itemize}

\begin{LONG}

\section{\sc Expert witness}
\textit{State of Idaho v. Gabriel L. Moreno and Anthony C. Moreno}, 2018 \\
Case Nos. CR-2017-8408-FE and CR-2017-8409-FE, District Court, County of Bannock, Idaho \\
Nature of Case: Second degree murder.
\par \hangindent=0.7cm \parindent=0.7cm \parskip=0.0cm Plaintiff alleged malice aforethought in a fistfight resulting in a death. The defendant claimed self-defense. The event was captured on video and posted to Snapchat, which became the primary exhibit. I testified regarding the source and analysis of the video.

%\section{\sc Teaching}
\section{\sc Courses taught}

%Courses taught
%\vspace{2mm}
%\begin{itemize}[label={},leftmargin=5mm]
\begin{itemize}[label={},leftmargin=0mm]
  \setlength\itemsep{0em}
%  \item CS 6150 Advanced Algorithms (with Valerio Pascucci, Fall 2013)
%  \item CS 354 Computer Graphics (Spring 2013)
%  \item CS 354 Computer Graphics (Fall 2012)
%  \item CS 108 UNIX (Fall 2011)
%  \item CS 105 C++ (Spring 2011)
%  \item CS 105 C++ (Fall 2010)
  \item Graduate Algorithms
  \item Computer Graphics
  \item Compilers
  \item Operating Systems
  \item Algorithms and Data Structures
  \item Introductory Programming
\end{itemize}

%Teaching assistant
%\vspace{2mm}
%\begin{itemize}[label={},leftmargin=5mm]
%  \setlength\itemsep{0em}
%  \item CS 312 Introduction to Programming (Summer 2012)
%  \item CS 354 Computer Graphics (Spring 2010)
%  \item CS 303E Elements of Programming (Fall 2009)
%\end{itemize}

\section{\sc Students}
\begin{itemize}[label={},leftmargin=0pt]
  \setlength\itemsep{0em}
  \item John Motley (current)
  \item Galen Cochrane (current)
  \item William Grigg (current)
  \item Joseph Valentin (current)
  \item Sandro Pawlidis (current)
  \item Jacqueline Ory (NIATEC)
  \item Nathan Morrical (PhD student, University of Utah)
  \item Marko Sterbentz (Masters student, University of Southern California)
  \item Jonathan Glines (NVIDIA)
  \item Nicholas Harrison (Clearwater, Inc.)
  \item Zackary Hall (Clearwater, Inc.)
%  \item Zinnia Mukherjee (Masters, University of Utah)
%  \item Sidharth Kumar (Postdoc, University of Utah)
%  \item Laura Lediaev (PhD, University of Utah)
%  \item Eric Greg Daniel (Google)
\end{itemize}

\section{\sc Collaborations}
Computer Science
\vspace{2mm}
\begin{itemize}[label={},leftmargin=5mm]
  \setlength\itemsep{0em}
  \item Kevin Parker (Idaho State University)
  \item Jonathan Holmes (Idaho State University)
  \item Valerio Pascucci (SCI, University of Utah)
  \item Christopher Johnson (SCI, University of Utah)
  \item Chandrajit Bajaj (University of Texas)
  \item Wenping Wang (University of Hong Kong)
  \item Peer-Timo Bremer (Lawrence Livermore National Laboratories)
  \item Attila Gyulassy (SCI)
  \item Brian Summa (SCI, University of Utah)
  \item Josh Johnston (Boise State University)
  \item Parris Egbert (Brigham Young University)
\end{itemize}

\vspace{2mm}
Other disciplines
\vspace{2mm}
\begin{itemize}[label={},leftmargin=5mm]
  \setlength\itemsep{0em}
  \item DeWayne Derryberry (Statistics, Idaho State University)
  \item Teri Peterson (Statistics, Idaho State University)
  \item Ken Aho (Biology, Idaho State University)
  \item Justin Stover (History, Idaho State University)
  \item Steven Shropshire (Physics, Idaho State University)
  \item Boyd Edwards (Physics, Utah State University)
  \item Donna Delparte (Geosciences, Idaho State University)
  \item Andrew Gillette (Mathematics, University of Arizona)
  \item Terrence Sejnowski (Neuroscience, Salk Institute)
  \item Tom Bartol (Neuroscience, Salk Institute)
  \item Kristen Harris (Neuroscience, University of Texas)
  \item Justin Kinney (Neuroscience, Massachussetts Institute of Technology)
  \item Daniel Johnston (Neuroscience, University of Texas)
\end{itemize}

\end{LONG}

\end{resume}
\end{document}

