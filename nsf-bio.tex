%%%%%%%%% BIOGRAPHICAL SKETCH -- 2 pages

\documentclass[svgnames,12pt]{article}
\usepackage[margin=1in]{geometry}
\usepackage[T1]{fontenc}
\usepackage{times}
\usepackage{url}
\usepackage{booktabs}
\renewcommand\thesection{(\alph{section})}
\renewcommand\thesubsection{(\alph{subsection})}

\newenvironment{tightenumerate}{
\begin{enumerate}
  \setlength{\itemsep}{1pt}
  \setlength{\parskip}{0pt}
  \setlength{\parsep}{0pt}
}{\end{enumerate}
}
\newenvironment{tightitemize}{
\begin{itemize}
  \setlength{\itemsep}{1pt}
  \setlength{\parskip}{0pt}
  \setlength{\parsep}{0pt}
}{\end{itemize}
}

\begin{document}

\begin{center}
\begin{large}\textbf{Biographical Sketch} \\
John M. Edwards, Assistant Professor \\ \end{large}
Utah State University \\
john.edwards@usu.edu \\
%(208) 282-4876 \\
\end{center}

%\textbf{(a) Professional Preparation} \\
\subsection*{Professional Preparation}
\begin{tabular}{llll}
Institution & Location & Major & Degree, Year \\
\midrule
Utah State University & Logan, Utah & Computer Science & B.S., 1998 \\
Brigham Young University & Provo, Utah & Computer Science & M.S., 2004 \\
The University of Texas & Austin, Texas & Computer Science & Ph.D., 2013 \\
The University of Utah & Salt Lake City, Utah & Visualization & Postdoc, 2013-2015
\end{tabular}

\subsection*{Appointments}
\begin{tabular}{lll}
Period & Appointment & Institution \& location \\
\midrule
2018-Present & Assistant Professor & Utah State University, Logan, UT \\
2015-2018 & Assistant Professor & Idaho State University, Pocatello, ID \\
2012 & Visiting Scholar & University of Hong Kong, Hong Kong, China \\
%2010-2013 & Assistant Instructor & The University of Texas \\
2008-2009 & Visualization Research Engineer & Autonomous Solutions, Inc., Logan, UT \\
2005-2008 & Software Engineer & ProLogic, Inc., Fairmont, WV \\
1999-2005 & Software Engineer & Rigaku, Inc., Houston, TX
\end{tabular}

\subsection*{Products}
\paragraph{Selected Products}
\begin{tightenumerate}
\item John Edwards, Erika Fulton, Jonathan Holmes, Joseph Valentin, David Beard, and Kevin Parker. Separation of syntax and problem solving in Introductory Computer Programming. IEEE Frontiers in Education. San Jose, CA. October 2018.
\item Boyd Edwards and John Edwards. Dynamical interactions between two uniformly magnetized spheres. European Journal of Physics, 38(1):015205, 2016.
\item Xin Tong, John Edwards, Chun-Ming Chen, Han-Wei Shen, Christopher Johnson, and Pak Chung Wong. View-dependent streamline deformation and exploration. \textit{IEEE Transactions on Visualization and Computer Graphics}. 22(7):1788-1801. 2016.
\item \emph{Phanon} - Computer programming educational software.\\
\url{http://phanon.herokuapp.com}
\item \emph{MagPhyx} - Magnet simulation software\\
\url{http://edwardsjohnmartin.github.io/MagPhyx}
\end{tightenumerate}

\newpage

\paragraph{Additional Products}
\begin{tightenumerate}
\item DeWayne Derryberry, Ken Aho, John Edwards, and Teri Peterson. Model selection and regression t-statistics. \textit{The American Statistician}. In press. 2018.
\item Boyd Edwards and John Edwards. Periodic nonlinear sliding modes for two uniformly magnetized spheres. Chaos: An Interdisciplinary Journal of Nonlinear Science, 27(5):053107, 2017.
\item Nathan Morrical and John Edwards. Parallel quadtree construction on collections of objects. \textit{Computers and Graphics}. 66:162168. 2017.
\item John Edwards, Eric Daniel, Valerio Pascucci, Chandrajit Bajaj. Approximating the Generalized Voronoi Diagram of Closely Spaced Objects. \textit{Computer Graphics Forum}. 34(2):299-309. 2015.
\item John Edwards, Eric Daniel, Justin Kinney, Terrence Sejnowski, Tom Bartol, Daniel Johnston, Kristen Harris, and Chandrajit Bajaj. VolRoverN: Enhancing surface and volumetric reconstruction for realistic dynamical simulation of cellular and subcellular function.  \textit{Neuroinformatics}. 12(2):277-289.  2014.
%\url{http://www.cs.utexas.edu/$\sim$bajaj/cvcwp/?page\_id=2089}
\end{tightenumerate}

\subsection*{Synergistic Activities}
\begin{tightenumerate}
\item \textbf{Program committee:} International Conference on Geometric Modeling and Processing (GMP) 2015, 2016, 2017, 2018. \textbf{Reviewer:} \textit{ACM SIGCSE 2019}, \textit{IEEE Frontiers in Education 2018}, GMP 2015, 2016, 2017, Computing Surveys, Computer Aided Geometric Design, European Symposium on Algorithms 2014, International Meshing Roundtable, SIGGRAPH Asia 2015. \textbf{Session chair:} Idaho Academy of Science and Engineering Annual Meeting, 2016.
%\item \textbf{Neuroscience:} Attended competitive NIH BRAIN Initiative Summer Course on interdisciplinary computational neuroscience, July 2016.
%\item \textbf{Geoscience:} Contributor in Managing Idaho's Landscapes for Ecosystem Services (MILES), an NSF-funded project to understand social and ecological systems.
\item \textbf{Outreach:} Acted as faculty advisor to student-led Google igniteCS high school outreach workshop series to Blackfoot High School, Spring 2017. Faculty advisor to Google ig- niteCS/Idaho STEM Action Center middle school outreach workshop series to eight local middle schools, Fall 2017.
\item \textbf{Teaching:} Development and delivery of undergraduate and graduate courses on data visualization, data science, advanced algorithms, graphics, compilers, operating systems. 2013-2019. Teaching Innovation Grant for Introductory Computer Programming course, 2017.
\item \textbf{Advising:} Joseph Ditton (Master's) - CS1 education. Selected undergraduate student projects: Nathan Morrical (now at UofU) - Parallel GVD; Marko Sterbentz (now at USC) and Galen Cochrane - Lidar data collection and analysis; Jonathan Glines (now at NVIDIA) - Bird flocking analysis.
\item \textbf{Applicable workshop:} NSF Developing Empirical Education Research Studies (DEERS). Charlottesville, VA. 3rd cohort. July 17-19, 2018.
%\item \textbf{NSF EPSCoR MILES mentor:} - supervised three students over two years to develop cutting edge Lidar collection, visualization and analysis software. (Award No. IIA-1301792)
\end{tightenumerate}

\end{document}
